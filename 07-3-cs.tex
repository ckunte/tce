\section{Banni Villages in Kutch} % (fold)
\label{sec:bvk}

Ever since man first made conscious attempts to create a proper environment to live in, he has been governed by a stronger urge to express himself in a way it is the same spirit that inspires him to adorn his own self from whatever is available around him. Man has shaped his habitat through experience and learning, which in turn has changed over time. To adorn and embellish his dwelling has been as much a part of his creative nature as it has been his desire to decorate his costumes and other objects of everyday use.

The traditional Banni villages are the best examples of such creativity. They show man's response to nature.

\subsection{Location} % (fold)
\label{sub:bvk_loc}

Kutch is an ancient land located in the Northwest region of Gujarat, the western most state in India with a rich history. The central belt is known as Banni. There are no urban settlements in Banni. In fact, not even villages are found. However, there are forty beautiful semi-nomadic hamlets.

[Image of Banni area]

Banni is a sparsely populated region. For ages, the people of this region has been involved in traditional occupation and lifestyle, and very limited education, if any at all even in the recent years. They are courteous and soft-spoken. About ninety percent of the population is Muslim, mostly of Jat origin. People from different caste groups do not occupy the same cluster.

Often clusters are separated by a few hundred metres. The communication between different clusters of caste groups is limited in part due to hostilities. A bush fencing is erected to define a cluster but this is not always the case.

There are no roads connecting the Banni hamlets, but as the land is flat, one may drive anywhere. There are some camel routes.

Most hamlets in Banni form thematic patterns in the way they are organised. Despite the differences in their play types, they follow a strong theme underlining type, construction form, and other elements. Materials used in their construction are limited to clay, wood and thatch, and are therefore relatable in character throughout these settlements.

In Banni, the scrubby bareness of surroundings is to a degree compensated for human effort, which makes best use of meagre resources. The stark initial impressions of conical thatched roofs standing rather low against the skyline is reversed dramatically when looked closely at a cluster of houses, which is uniquely indigenous.

% subsection bvk_loc (end)

\subsection{Planning} % (fold)
\label{sub:bvk_planning}

The spatial organisation of the Banni dwellings does not conform to any predetermined or consciously laid-out street pattern. It is more an act of spontaneous, organic growth, depending upon the needs of the community. However, despite its apparent randomness, there is a clear distinction of territorial claims and the right of way. The meandering paths are not erratic, but are defined by the edges of the platforms, which indicate dwelling spaces, both internally as well as externally. The bordering houses are typically oriented away from pathways in order to ensure privacy.

The clay floor-scape, developing into varied platforms, rises vertically to form the walls of circular houses. The walls terminate in pure conical thatch roofs. The roof overhangs, projections, and shade sculptural walls, which are sometimes adorned by colourful geometric and floral patterns.

A round hut house as a dwelling raised on a platform is the most common element in this area. This round hut is known as a 'Bhunga' locally. Varying from 3--6m in diameter, it is the main habitable space. The rectangular hut is called a 'Choki'. The larger ones are used as living spaces, while the smaller are for cooking. Those units smaller than Chokis, about 1.5m in height and without roofs are often not definite in shape, but are often attached to Bhungas. These are used for washing, bathing, and as storage units. The bathing and washing areas are limited on account of water scarcity. Since sun-dried clay blocks are the major components of construction, circular shape has been met with greater success due to its response to compressive forces. This has resulted in the greater use of Bhungas as major dwelling spaces.

Different spaces, circular or otherwise, are not interconnected due to their potential for cracking at the junctions. A gap is always left between two major built-up areas to allow for some expansion. A horizontal platform in clay of about 50cm high, however, connects up all the features.

% subsection bvk_planning (end)

\subsection{Construction} % (fold)
\label{sub:bvk_const}

A Bhunga enclosed by a mud wall is the most typical construction for dwelling purposes. Its diameter may vary from 3--5m. The wall is usually constructed in two ways depending upon its location. In places that are unlikely to face inundation from rains, the walls are made of sun-dried clay blocks and are finished with mud plaster. These walls can neither carry the load of the roof, nor are they rigid enough to hold it. The roof load therefore is cleverly transferred about head level. A wooden prop placed in the centre of the beam supports the conical roof and helps transfer the load to the posts through beams. Often, two posts carrying the beams are placed outside the circular wall and are left exposed. At times they are embedded in the mud wall.

[Image here: Rising flood-scape through platform to mud walls to the thatched roof]

Alternatively, in areas that are likely to encounter water-logging, Bhungas are built with wooden sticks and covered in mud plaster. In the event of inundation, the wall would not giveaway, as it would be reinforced by the wooden sticks and is really a kind of adobe construction. These reinforced walls have greater load-bearing capacities, thereby eliminating the need for additional posts and beams. For the roof, a conical frame made of wood is filled with sticks.

All interior spaces are finished with white clay, often of good quality. People from communities involved in crafts, as in Ludia, finish these interiors in patterned clay and mirrors are embedded flush with the surface. These mirror designs are well integrated with the patterns on walls. Small granaries are also made of clay. These are decorated to match the interiors, or at times left plain. The granaries may be either circular or rectangular in shape. Clay is also used for making chests with wooden shutters to take care of other storage needs.

[Image here: Intricate clay embellishment on interior walls]

Jhompas, which are single-cell spaces, are also constructed with mud as the principal material. However, they are variations in the construction of the walls. Besides using clay blocks, there are also places where the walls are first made with sticks and then finished with mud. These two methods are comparable with the construction of Bhungas. The third method uses straw for the walls. Here the straw is tied with ropes and raised in a circular form. The ropes are made from fibrous local plants. These thick ropes run horizontally along the internal and external surfaces of the straw walls, which are then finished with clay and cow-dung. A shallow foundation of about 30--40cm is made for securing the Jhumpa.

The roof is supported in two different ways, depending upon the nature of wall construction. While stronger mud block walls and the walls reinforced with wooden sticks can support the cone of the roof on its periphery, the straw reinforced walls normally have a central wooden spar supporting the apex of the cone. Radial supports of raw wood connect the spar with the periphery. These are then tied with ropes and the voids are filled with straw and covered by thatch. In the other method, radiating supports are directly connected at the apex, tied with ropes and then roofed on. In such cases, the walls are also provided with hoops are regular intervals. Thus the hoops take care of the tension caused by the thrust of the conical roof. Sometimes a mud wall is only a portion and the roof is supported on wooden posts. The roof projects all along the periphery to protect the walls from the sun and rain.

The rectangular spaces are generally constructed with sun-dried clay blocks, making the wall 40--50cm thick. Raw wood joists span these narrow and long rooms. The spans are further reduced by wooden sticks put across on the main members. These roofs are finished with rammed earth. One of the roofs is for storing grass and straw, however, in such cases the load is taken on wooden posts erected outside the mud walls.

% subsection bvk_const (end)

\subsection{Features} % (fold)
\label{sub:bvk_feat}

Paintings and drawings on walls or on the ground in front of the houses are a part of ritual culture in villages all over India. Often simply a visual complement to a complex social sacrament, they nevertheless express an integral aspect of rural architectural ornamentation. They are usually associated with ritual cleaning of the house in some places every morning before sunrise, while others in preparation for an important event.

[Image here: Painting the walls is a ritual here]

Most villages in India still maintain some form of artistic continuity and a symbiotic wit of their past, and styles have been adapted locally to suit both historic and contemporary influences. In larger towns and cities, these traditions have for the most part been lost.

Regional variations in decoration parallels those that exist in built forms. The influence of climate, culture, and other local characteristics also plays a significant role. Decoration can also be related to the extensive use of jewellery and boy ornamentation.

When a community uses clay for making homes, it also uses it for making other objects of everyday use. Like in the North-eastern parts of India, where bamboo is the principal material, its application can be seen in every object created within living environment. Landscape that is rich in flora, like in Kashmir, has generated more floral and organic forms and patterns in man-made objects. Whereas desert areas resort to stronger and purer colours with geometric patterns.

[Image here: Geometric forms dominate the embellishment]

Geometric forms dominate nearly all of architecture and craft activities in villages of the desert. Even the floral patterns are geometric in design. The greatest of all virtues is the manner in which these expressions pervade and link activities like house construction, clay products, woodwork, and textile crafts.

Local ponds and ditches in the vicinity of the village are the main sources of clay. Everyone from the village has a claim on these spots. Most of the clay work is done by women who transfer this skill to their daughters and keep the tradition alive. The training begins early in childhood, and they acquire proficiency by the time they reach maturity. However, the difficult part of the job is done by men who dig and help in carrying the clay from its source to the work site.

The plain clay surfaces on the exterior of these built forms provide a contrast with the interior decorative relief work done. Many of these houses have this decorative relief, though these vary in magnitude and quality. Alcoves of different sizes are moulded harmoniously with decorative relief work to create clay shelves for storage. There are many other clay objects which form part of the living space. Often, these objects are also decorated with similar relief work in clay.

% subsection bvk_feat (end)

% section bvk (end)
