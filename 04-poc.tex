\chapter{Preservation of culture} % (fold)
\label{cha:poc}

Culture is defined in many ways. It is:

\begin{itemize}
  \item The art and other manifestation of human intellectual achievement regarded collectively
  \item A refined understanding of this intellectual development
  \item The customs, civilisation and achievement of a particular time or people
  \item Improvement by mental or physical training
\end{itemize}

\noindent Thus culture is one of the most precious possession of any civilisation. And various art forms constitute the soul of this culture. It is through various arts and traditions and crafts that any culture preserves its existence and flourishes.

\section{Present scenario} % (fold)
\label{sec:ps}

In the words of Pandit Jawaharlal Nehru, The discovery of India -- what I have discovered. 

"India is a geographical and economical entity, a cultural unity admits diversity, a bundle of contradictions held together by strong but invisible threads. Overwhelmed again and again, her spirit was never conquered.

About her there is the elusive quality of a legend long ago; some enchantment seems to have held in her mind, she is a myth and an idea, a dream and vision, and yet very real, present and persuasive.

But however she changes that old witchery will continued and hold the hearts of her people. Though her attitudes may change, she would continue as of old and her store of wisdom will help her to hold on to what is true and beautiful and good in this harsh vindictive and graspic world."

Each and every corner of this large country has its distinctive culture, identity and so many things to offer right from various times of performing arts to various arts of painting, weaving, carving, pottery, and what not. Every artist in his field tries to work hard and using his ancestral gifted skills or may be those achieved through his own efforts to carry over that art form. Through this he also preserves his culture. But mainly he is doing that to earn his livelihood.

Great designer Ettore Sottsass says about Indian craftsmen:

"Indian craftsmen are open minded and totally concentrated on the tiny operations they do with their hands, on the silent, magic, surprising rituals performed by their finger, but on the contrary, they are also unaware and removed from the storms, the temptations of the world."

For every art and artist it is the most important thing whether it gets some patronage or not. The art which was in its golden form when it was having royal patronage is dying today due to lack of proper appreciation and the returns which is it worth of. Due to this reason, which an artist finds that his father doesn't earn the required, inspite of so much of hardwork, naturally he finds some other way to work. Because it is his livelihood that it is more important. In this process, the craftsmen are increasingly losing touch with their own traditions in terms of materials, techniques, designs and aesthetics of their arts and crafts. Sudden changes caused by modern industrialisation is also one of the reasons behind it.

But today with rapid rise in living standard of middle class and higher middle cast, they have raised their hearts for these art objects. A man wants his drawing room to be adorned with a beautiful mural, white a lady wants her wardrobe to be fulfilled with traditional patolas and paithanis, her bedspread to be nicely embroidered and her dining to display various antique pieces.

But these changing trends are still not sufficient to rejunivate the crafts industry and most of the artists are still there where they were. The reason behind this being the presence of a middle man or a middle agency. Such middle person provides the urban consumers with their requirements from the artists. But keeps the major share of profit for himself. Thus the artist gets much less than what he should really be getting.

% section ps (end)

\section{Adapting to changing trends} % (fold)
\label{sec:act}

Adaptability is the hallmark of a real craftsman. Machine can imitate an artist, but the craftsman has to be always one step ahead of it. One of the Zardozi and Sequins craftsmen says, "How can I ignore changes in materials, tools, and usages? Where are the elephants, kings and palaces today to decorate with traditional items? The instant feedback, as in fashion is what one hungers for more and more and that is what keeps us on our toes!

This is very true. Too much running after old things is dangerous. The situation today is changing. The artist should find out and understand what really a buyer would buy. Accordingly he should find out and understand what really a buyer wants and should do the innovations. His beautifully embroidered traditional ghagras and jackets won't fetch much value in market, but with changing trends if he does that embroidery on salwar suits or bedspreads, he would get much better returns.

For this the craftsman has to be sensitive to and aware of the current trends and should be able to put his traditional skills to the modern use.

% section act (end)

\section{Revitalisation of traditional art forms via tourism} % (fold)
\label{sec:rev}

Tourism development stimulates the revitalisation of traditional art and hosts and impetus new creations. It provides market that has helped to preserve traditional art forms and keep culture alive.

Tourism increases the demand of arts and crafts with the result there is an abundance of jewellery, baskets, handicrafts, silver works, pottery, etc. Explorations by new ethnic groups has resulted in the borrowing, modification and refinement and development of particular art farm of art.

In brief tourism leads to:

\begin{enumerate}
  \item Continued production of art forms by the tribals
  \item Production of different art forms
  \item Maintenance of strong symbolic value of the product
  \item Availability of trained craftsmen
  \item Improvement in quality and artistic designs of arts and crafts
  \item Production of art form based on traditional ancestral lifestyle
  \item Expression of culture through art forms
  \item Creation of employment is fabrication of arts and crafts
  \item Renaissance in the production of art forms by avoiding the trends of specialisation standardisation and simplification
\end{enumerate}

% section rev (end)

% chapter poc (end)
