\chapter{Introduction} % (fold)
\label{cha:intro}

% Strong desires drive humans. Among these, the one to see places, understand one another, and experience culture is high on the list. Tourism enables this. In fact, 

Culture and heritage is the essence of tourism at many destinations in the world. It is a form of identity, new and novel, that people go in search of. Every year, thousands travel to Rome to imagine gladiators in the \emph{Colosseum}, to Beijing to imagine man's capacity to build and to feel the exhilaration of being on the \emph{Great Wall}, walk the way of \emph{Camino de Santiago} to experience the divine, or to Agra to see the extent of expressed love in the form of \emph{Taj Mahal}.

Although world heritage sites draw tourists both domestic and international alike in large hordes, most of world's historic sites are not internationally known and only relatively few ever attract international tourists, except perhaps in combination with other attractions. For every world renowned cultural attraction, there are thousands more that await discovery.

\section{How tourism flourishes} % (fold)
\label{sec:htf}

For all the money in the world, the best and the fastest modes of transport are of no use unless man has the time. It should be possible to get away from preoccupation for a reasonable length of time to able to go see places. A few decades ago, paid holidays were few and far in-between. Moreover, one seldom took (or allowed) holidays at a stretch, and short ones were not conducive.

The situation though is steadily improving. Employers are realising that work without respite stifles efficiency and productivity. This is aside from the fact that it is leisure, and not work, that has been man's natural occupation. Industrialisation has increased the living standards of ordinary mass of people. In turn, this has allowed more time for the leisurely pursuits.

Tourism is a form of mobility for leisure. It is also the world's fastest growing industry societies indulge-in. Cohen defines the tourist as ``a voluntary, temporary traveller, travelling in the expectation of pleasure from the novelty and change experienced on a relatively long and non recurrent round-trip.''

% section htf (end)

\section{Tourism as an industry} % (fold)
\label{sec:tai}

Tourism is a great source of economic growth and environmental development. It has emerged as an organised sector balancing demand and supply like an industry. It generates wealth and employment, enables making places of interest, and its people prosperous. It enables renting natural resources, e.g., beaches, moors, deserts, etc., which may otherwise offer no economic value by themselves.

This is an industry that generates 3.5 trillion dollars in annual revenue, employing 112 million people, and delivering 5.5\% of the world's \textsc{gdp}. It is the world's second largest only after the energy. But one without fire or smoke, while enabling (a) education without classrooms, (b) integration sans legislation, and (c) diplomacy without formality. With continuous improvement, creating facilities and attractions, it has not stood still. Never have there been as many choices for the tourist. Innovations such as time-sharing and activity vacations are designed to provide value through fair-priced accommodations and purposeful travel options. Learning a foreign language or a new skill while relaxing in pleasant surroundings offers the means to justify and value vacation. With such trends, futurists see an optimistic future for the travel industry.

Today, the customer is not only expansive in preferences, but is also demanding. This for the greatest joy possible in the shortest possible time. This has changed perspective and pattern of the tourism industry. And it has warranted a multidimensional approach to please the varied interests of the consumers, a diverse mosaic of personalities, fanciful in their estimates and desires. All this requires constant planning, monitoring, continuous creation of the new and exotic, while making sustained efforts to keep the interest alive.

% section tai (end)

\section{Social aspects of tourism} % (fold)
\label{sec:sat}

Tourism offers many intangible benefits, viz., in educational, social, cultural, and political sectors. It generates international recognition. Travel widens knowledge and reduces political sources of friction and tensions. This applies very much to India, being a vast and diverse country with rich regional, lingual, and cultural heritage. Just as movement of people between regions has helped in the past in terms of multiculturalism, shared values, etc., so will it in shaping her future into an integrated nation.

% section sat (end)

\section{Tourism and cultural heritage} % (fold)
\label{sec:tch}

Culture can best be defined as the behaviour as observed through social relations and natural artefacts. In anthropology, culture includes patterns, norms, and standards, which find expression in behaviour, social relations, and artefacts.

Culture is a great motivation for travel. The presence and activities of tourism change the elements of culture. The frequently criticised tourism industry has been winning recognition for its role in saving many of the world's cultural treasures. Along with a reason to preserve traditions and historical landmarks, it also is credited for keeping local craft industries alive. Many like ancient weaving and wood carving techniques have been saved by tourism.

% section tch (end)

% chapter intro (end)
