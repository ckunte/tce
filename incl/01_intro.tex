\chapter{Introduction} % (fold)
\label{cha:intro}

Culture and heritage are often the heart and soul of tourism in many destinations worldwide. They represent a unique and novel form of identity that people seek. Year after year, thousands of tourists travel to Rome to imagine gladiators in the \emph{Colosseum}, to Beijing to marvel at mankind's ability and to experience the awe-inspiring feeling of walking the \emph{Great Wall}, to embark on the \emph{Camino de Santiago} to experience the divine, or to Agra to witness the gravitas of expressed love through the exquisite \emph{Taj Mahal}.

World heritage sites attract both domestic and international visitors in large numbers, but most of the world's historic sites are not well-known. Only a relatively small number of historic sites attract internationally, often in combination with other attractions. In fact, for every globally recognised cultural attraction, there are thousands more that await discovery.

\section{How tourism flourishes} % (fold)
\label{sec:htf}

No matter how luxurious or speedy the mode of transportation may be, it is all for naught if one does not have the time to enjoy the journey. It is essential to take a break from daily preoccupations and escape to explore new places. In the past, paid holidays were a rare luxury, and it was uncommon to take extended time off away from work. Things have improved gradually though, and more and more employers now support improved work-life balance by intent as well as by policy. After all, leisure time is just as important as work, and it is essential for one's well-being. Thanks to industrialisation and rising living standards, people have more opportunities now to pursue their leisurely passions.

In the past, overwork was seen as a badge of honour, but employers today are recognising the upside of balancing workload, owing to the benefits, viz., improved productivity, mental well-being, and also in retaining talent. This has allowed individuals to pursue hobbies, travel, and engage in other leisurely pursuits, as after all, leisure has been a part of human life for centuries. This is a positive development that benefits both employees and employers.

Tourism can be a way for people to explore new places and enjoy leisurely time. It is also a fastest-growing industry worldwide. According to Cohen, a tourist is someone who travels voluntarily and temporarily, seeking pleasure from experiencing something new and different during a relatively long, one-time trip.

% section htf (end)

\section{Tourism as an industry} % (fold)
\label{sec:tai}

Tourism is a great source of economic growth, for developing and improving the environment. It has emerged as an organised sector that balances supply and demand like an industry. It enables renting natural resources, e.g., beaches, moors, deserts, etc., which may otherwise offer no economic value by themselves. So it is fair to say that tourism generates wealth and employment, brings prosperity to places of interest, and its people. 

This is an industry that generates 3.5 trillion dollars in revenue annually, employing 112 million people, and delivering 5.5\% of the world's \textsc{gdp}. It is the world's second largest only after energy, enabling (a) education without classrooms, (b) integration sans legislation, and (c) diplomacy without formality. With continuous improvement, creating facilities and attractions, it has not stood still. Never have there been as many choices for the tourist. Innovations such as time-sharing and activity vacations are designed to offer value through fair-priced accommodations and purposeful travel options. Learning a foreign language or a new skill while relaxing in pleasant surroundings offers the means to justify and value vacation. With such trends, futurists see an optimistic future overall for the travel industry.

Customers today have expansive preferences and demands. They seek the greatest enjoyment possible in the shortest amount of time, which has driven the industry to shift its perspective and approach. Meeting the diverse and sometimes fanciful interests of consumers requires a multidimensional approach, with continuous planning, monitoring, and creation of new and exotic offerings to keep interests alive.

% section tai (end)

\section{Social aspects of tourism} % (fold)
\label{sec:sat}

Tourism offers many intangible benefits, viz., in educational, social, cultural, and political sectors. It generates international recognition. Travel widens knowledge and reduces political sources of friction and tensions. This applies very much to India, being a vast and diverse country with rich regional, lingual, and cultural heritage. Just as movement of people between regions has helped in the past in terms of multiculturalism, shared values, etc., so will it in shaping her future into an integrated nation.

% section sat (end)

\section{Tourism and cultural heritage} % (fold)
\label{sec:tch}

Culture is a fascinating and complex concept that encompasses a wide range of human behaviour, social interactions, and physical artefacts. In anthropology, it is defined as a set of patterns, norms, and standards that shape our beliefs and actions, and can be observed in the way we relate to others and the world around us. From language and religion to art and technology, culture plays a crucial role in shaping our identity and the way we experience life. By studying it, we can gain a deeper understanding of the diverse and intricate ways in which humans express themselves and interact with one another.

Travel is often motivated by a desire to experience different cultures and explore new ways of life. Culture in turn is shaped by the presence and activities of tourists, creating a of ever-evolving relationship between guests and hosts. Despite the criticism the tourism industry has received in the past, it is now being recognised for its role in preserving heritage and supporting local craftsmanship. From ancient weaving techniques to traditional wood carving, many cultural practices and traditions have been saved, even revitalised thanks to the interest and investment from tourists. By engaging with local communities and respecting their cultural traditions, travellers can help support sustainable and responsible tourism practices that benefit both and the destinations they visit.

% section tch (end)

% chapter intro (end)
