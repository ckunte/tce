\chapter{Introduction} % (fold)
\label{cha:intro}

Culture and heritage are often the heart and soul of tourism in many destinations worldwide. They represent a unique and novel form of identity that people seek out. Every year, thousands of tourists travel to Rome to envision gladiators in the \emph{Colosseum}, to Beijing to marvel at mankind's architectural prowess and to experience the awe-inspiring feeling of walking the \emph{Great Wall}, to embark on the \emph{Camino de Santiago} and encounter the divine, or to Agra to witness the power of expressed love through the exquisite \emph{Taj Mahal}.

World heritage sites attract both domestic and international tourists in large numbers, but most of the world's historic sites are not well-known internationally. Only a relatively small number of historic sites attract international visitors, often in combination with other attractions. In fact, for every globally recognized cultural attraction, there are thousands more awaiting discovery by curious travellers.

\section{How tourism flourishes} % (fold)
\label{sec:htf}

No matter how luxurious or speedy the mode of transportation may be, it is all for naught if one does not have the time to enjoy the journey. It's essential to take a break from daily preoccupations and escape for a while to explore new places. In the past, paid holidays were a rare luxury, and it was uncommon to take extended time off work. However, things have gradually improved, and more and more employers now realize that overworking their employees can actually reduce productivity. After all, leisure time is just as important as work, and it is essential for our well-being. Thanks to industrialization and rising living standards, people now have more opportunities to pursue their leisurely passions.

In the past, work without respite was often seen as a badge of honour, but employers today are beginning to recognize the detrimental effects of overworking their employees. It is not only inefficient but also goes against our natural instincts. After all, leisure has been a part of human life for centuries. With increased industrialization and higher living standards, people have more leisure time at their disposal. This has allowed individuals to pursue hobbies, travel, and engage in other leisurely pursuits. Overall, this shift in attitudes towards work and leisure is a positive development that benefits both employees and employers.

Tourism can be a way for people to explore new places and enjoy leisure time. It is also one of the fastest-growing industries worldwide. According to Cohen, a tourist is someone who travels voluntarily and temporarily, seeking pleasure from experiencing something new and different during a relatively long, one-time trip.

% section htf (end)

\section{Tourism as an industry} % (fold)
\label{sec:tai}

Tourism is a great source of economic growth and environmental development. It has emerged as an organised sector balancing demand and supply like an industry. It generates wealth and employment, enables making places of interest, and its people prosperous. It enables renting natural resources, e.g., beaches, moors, deserts, etc., which may otherwise offer no economic value by themselves.

This is an industry that generates 3.5 trillion dollars in annual revenue, employing 112 million people, and delivering 5.5\% of the world's \textsc{gdp}. It is the world's second largest only after the energy. But one without fire or smoke, while enabling (a) education without classrooms, (b) integration sans legislation, and (c) diplomacy without formality. With continuous improvement, creating facilities and attractions, it has not stood still. Never have there been as many choices for the tourist. Innovations such as time-sharing and activity vacations are designed to provide value through fair-priced accommodations and purposeful travel options. Learning a foreign language or a new skill while relaxing in pleasant surroundings offers the means to justify and value vacation. With such trends, futurists see an optimistic future for the travel industry.

Today's customers have expansive preferences and high demands. They seek the greatest enjoyment possible in the shortest amount of time, which has forced the tourism industry to shift its perspective and approach. Meeting the diverse and sometimes fanciful interests of consumers requires a multidimensional approach, with constant planning, monitoring, and the creation of new and exotic offerings to keep their interest alive.

% section tai (end)

\section{Social aspects of tourism} % (fold)
\label{sec:sat}

Tourism offers many intangible benefits, viz., in educational, social, cultural, and political sectors. It generates international recognition. Travel widens knowledge and reduces political sources of friction and tensions. This applies very much to India, being a vast and diverse country with rich regional, lingual, and cultural heritage. Just as movement of people between regions has helped in the past in terms of multiculturalism, shared values, etc., so will it in shaping her future into an integrated nation.

% section sat (end)

\section{Tourism and cultural heritage} % (fold)
\label{sec:tch}

Culture is a fascinating and complex concept that encompasses a wide range of human behaviour, social interactions, and physical artefacts. In anthropology, culture is defined as a set of patterns, norms, and standards that shape our beliefs and actions, and can be observed in the way we relate to others and the world around us. From language and religion to art and technology, culture plays a crucial role in shaping our identity and the way we experience life. By studying culture, we can gain a deeper understanding of the diverse and intricate ways in which humans express themselves and interact with one another.

Travel is often motivated by a desire to experience different cultures and explore new ways of life. Culture, in turn, is shaped by the presence and activities of tourists, creating a dynamic and ever-evolving relationship between visitors and locals. While the tourism industry has received criticism in the past, it is increasingly being recognized for its role in preserving cultural heritage and supporting local craft industries. From ancient weaving techniques to traditional wood carving, many cultural practices and traditions have been saved and revitalized thanks to the interest and investment of tourists. By engaging with local communities and respecting their cultural traditions, travelers can help support sustainable and responsible tourism practices that benefit both the visitors and the destinations they visit.

% section tch (end)

% chapter intro (end)
