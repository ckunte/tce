\chapter{Cultural tourism} % (fold)
\label{cha:ct}

`Tour' is a Hebrew word. It derives its meaning from the Hebrew term `Torah' which means learning, studying and search. Torah is the name given to Jewish law the book that defines Jewish way of life. A tour represents an attempt by the traveller to discover something about a place he visits. A tourist wants to see himself something he has heard about or wants to learn about.

The cultural effect of tourism has been a favourite subject of investigation among anthropologists. When cultures come into contact of any duration each becomes somewhat like the other through a process of borrowing.

Leisure is essentially the time available to individual when discipline of work, sleep and other basic needs have been met. Recreation covers boarding any pursuits to keep up during leisure time. Tourism overlaps with recreation as a way of using leisure and also other activities.

They are in many aspects subject but can be grouped into five broad categories:

\begin{enumerate}[noitemsep]
  \item Those taking place in and around home
  \item Cultural and artistic pursuits
  \item Activities with high social content (e.g. entertaining)
  \item Active pursuits of sports
  \item Other outdoor / indoor activities
\end{enumerate}

\noindent The first three of these although a major constituent of leisure time is outside the scope. Social and cultural interest are included only in so far as they are component of tourism.

\emph{Push Elements.} Factors which create internal desire to travel are push elements. Some of the important push elements are as under:

\begin{enumerate}[noitemsep]
  \item Family visit: Visiting relatives, friends, homeland, etc.
  \item Pleasure visit: For pleasure.
  \item Professional visit: Business travel.
  \item Curiosity: To experience new places, people, cultures, etc.
  \item Spiritual: For spiritual or religious purposes like pilgrimage, etc.
  \item Outdoor activity: To view or participate in sports, etc.
  \item Health: To health resorts, spas, etc.
  \item Image: `Impressing others' could be a reason enough to travel
\end{enumerate}

\noindent\emph{Pull Elements.} External factors which attract people to a particular place once the desire to travels there are known as pull elements. Some of the major pull elements are enumerated below:

\begin{enumerate}[noitemsep]
  \item Climate
  \item Architecture
  \item History, culture, tradition, etc.
  \item Cuisine
  \item Shopping
  \item Wild life
  \item `Collect' countries (e.g. bucket list)
  \item Love of travel
\end{enumerate}

\section{Types of tourism} % (fold)
\label{sec:types}

As we have seen, people travel for diverse reasons. Such objectives are sometimes used to designated the so called `types' of tourism. Although the categorisation appears artificial, it is useful in as far as it reveals various need of man being fulfilled by tourism. Some of the common objectives which prompts tourism are considered here under.

\begin{description}
  \item[Recreational:] This can either be related to sports (e.g. boating, camping, climbing, diving, fishing, hiking, hunting, river-rafting, para-gliding, skiing, swimming, et al) or can be forms of social or communal activities (e.g. beach-going, scouting, site-seeing, trekking, group events, visiting wild life sanctuaries, etc.). The mode of transport and particularly the choice of destination are usually determined by the form of relaxation desired. Such tourists indulge in their pursuits either individually or in small groups of people with common or similar interests.

  \item[Cultural:] Interests in diverse cultural patterns extant in different regions of the world, keenness to experience new cultural activities may constitute this category. Destinations depend upon the location and availability of events like folklore, art, music, fairs, etc. Where tourists have to go far to witness the cultural event of their choice they do expect reasonably good, frequent, and cheap transport or a well-organised coach service specially designed for the purpose.

  \item[Historical and Archaeological:] Interest in history is but natural. Given an opportunity, the desire to visit heritage locations (e.g. forts, museums, temples, mosques, cathedrals, monasteries, ruins, and excavation sites of bygone civilisations, etc) forms this type. Apart from comforting accommodations, availability for knowledgeable guides with good to reasonable language skills are sought after. Printed material, and preferably multilingual, on the monuments (e.g. brochures, handouts, photographs, picture postcards) are of great interest.

  \item[Education:] Quest for knowledge is a common enough characteristic. The destination of choice is obviously conditioned by the type of knowledge or skill sought. Those that travel for educational purposes usually to stay in university colleges or campuses that offer courses preferred.

  \item[Ethnic:] This involves contact with unusual and quaint customs of certain people usually inhabiting remote areas. The tribal folk and their unique customs and traditions, art and cultural forms draw the interested tourists to understand and sometimes experience the lifestyle.

  \item[Environmental:] The reason hill and forest resorts are popular is because of the health benefits they tend to offer. Clean air, cool weather, hot water springs, and unbound nature, away from the man-made hazard called the urban life, has the ability to lift spirits high and up, and is very attractive.

  \item[Business:] Business travel constitutes a seizable segment of modern tourism. Most business travellers need individual attention and arrangements. The target destinations are usually large cities offering business related attractions in the form of exhibitions, trade fairs, conferences, meetings, etc. Business tourism tends to be price inelastic and non-seasonal. Due to its highly individualistic character a lot of work has to be put in by the organisers but then it is usually proportionally profitable as well.

  \item[Exotic and unusual holiday:] As the name implies the exotic and unusual holiday tourism involves distant but sophisticated places. These demand superlative accommodation, cuisine and subsidiary services. Even if the trip is planned as an adventure holiday the tourists will accept what is offered yet they would want it to be in character with their vacation. Apparently it is the neo-rich classes which indulge in the exotic and unusual holiday tourism while the traditionally rich classes tend to be relatively more conservative.

\end{description}

\noindent Not withstanding the various options open to indulge in their particular whim or fancy, after all, `money makes the mare go'. Thus it is possible to correlate certain forms of tourism to specific income groups. The lower income groups along with a section of middle group generally tend to indulge in recreational tourism. Likewise, historical and ethnic tourism is largely supported by higher income brackets. Of course this is not to suggest any water-tight compartments. The present people of all income and educational levels go globe-trotting in pursuit of their particular need and desire.

Pleasure tourism is fast becoming a norm. Many societies consider it a symbol of their status. Not travelling overseas for holidays are viewed poorly, resulting in one's motivation to `improve' stature through the compulsive act of travel. Such tourists go in for a mode of transport and accommodation which they think is commensurate with the job status or social standing.

\section{Types of tourists} % (fold)
\label{sec:tot}

The basis on which typologies are created and purposes intended have given rise to distinctly different types of tourists. Smith separates tourists in seven categories:

\begin{description}
  \item[Explorer:] Very limited numbers looking for discovery and involvement with the local people.
  \item[Elite:] Special individually tailored visits to exotic places.
  \item[Offbeat:] The desire to get away from the crowds.
  \item[Unusual:] The visit with peculiar objectives such as physical danger or isolation.
  \item[Incipient mass:] A steady flow travelling along or in small organised groups using some shared services.
  \item[Mass:] The general packaged tour: Market leading to tourist enclaves overseas.
  \item[Charter:] Mass travel to relaxation destinations which incorporate as many standard western facilities as possible.
\end{description}

Smith's typology based on number tourists and their adaptation to local norms is shown in the following \autoref{tab:tt}.\cite{vsmith}

\begin{table}[H]
\caption{Types of tourists, their frequency, and their adaptations to local norms}
\label{tab:tt}
\centering
\small
  \begin{tabular}{+l^l^l}
  \rowstyle{\itshape}
  Type & Numbers & Adapting to local norms \\
  Explorer & Very limited & Accepts fully \\
  Elite & Rarely seen & Adapts fully \\
  Off-beat & Uncommon but seen & Adapts well \\
  Unusual & Occasional & Adapts somewhat \\
  Incipent mass & Steady flow & Seeks western amenities \\
  Mass & Continuous flow & Expects western amenities \\
  Charter & Mass arrivals & Demands western amenities \\
  \end{tabular}
\end{table}

%(Source: Smith Valene L., ed, Hosts and Guests: The Anthropology of Tourism, 1978 pg.)

\noindent The above table can roughly accommodate Cohen's classification due to common stress of the authors on the need for strangers and familiarity -- illustrated by Cohen in terms of their adaptation. However, according to Smith the most crucial factor in determining tourism impact is their numbers. Explorers and elite tourists by virtue of their limited numbers have limited impact on the indigenous culture. With increasing numbers the stressful effects are felt on the economic and socio-cultural systems. This is shown diagrammatically in the figure below.

[Add figure here from Smith Valene L., Tourist impact upon a culture]

% section tot (end)

% section types (end)

\section{Importance of cultural tourism} % (fold)
\label{sec:imct}

``There is no better bridge between people, ideas, ideologies, cultures than travel. It can nurture understanding with a country and between countries.'' [Powell, Travel Times, Section II, Times of India, 29 Apr 1988.]

In cultural tourism the tourists treat their stay in that destination as a rare opportunity to be away from inhibiting influence of the home discipline away from inhibiting influence of the home discipline and their particular lifestyle at home. They are out to enjoy such short lived freedom to the utmost and shall brook no interference with what they want. Mostly they are short on time and would like their desires, whims, and fancies to be fulfilled to the maximum extent.

Culture is a great motivation for travel. The presence and activities of tourism change the elements of culture. Cultural tourism can be defined as the ``Absorption by tourists of features resembling the vanishing style of past societies absorbed through such phenomena as house styles, crafts, farm equipment, and dresses.'' [Smith, V. (1977), Hosts and Guests, An Anthropology of Tourism, University of Pennsylvania Press, Philadelphia pp. 2-3]

The main elements of culture which attract tourists to a particular destination fall under the following categories:

\begin{itemize}[noitemsep]
  \item Traditional and festive
  \item Art \& music, festivals, carnivals, concerts, paintings and sculptures
  \item The theory of the region and historical buildings and elements
  \item Architecture
  \item Handicrafts
  \item Places of religious interests
  \item Educational systems
  \item Dresses
  \item Language
  \item Gastronomy
  \item Technology
\end{itemize}

\noindent Cultural tourism stimulates the revitalisation of traditional art of host and impetus new creations. It provides a market that has helped preserve traditional art forms and keeps the culture alive. It brings awareness in the artists and craftsmen about the skills within them and boasts a self confidence in them for better living.

Due to a perfect cultural tour the tourist gets maximum satisfaction of his tour having enjoyed and felt a lifestyle different than his own. That rejuvenates his tired mind due to busy urban routine. Thus the cultural tourism satisfies the needs of both -- the tourist and the host.

% section imct (end)

% chapter ct (end)
