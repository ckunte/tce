\chapter{Design solution} % (fold)
\label{cha:ds}

The problem of isolation of tourist and the local masses has reached an acute stage and needs to be dealt efficiently. Also the skills of the craftsmen should be chanellised in a proper way.

Architecturally, this problem can be tackled by providing the tourist, an environment where they will come in close contact with the life style and which in turn will inspire the tourist to take part in the way of life of common people. For this a tourist village instead of a hotel can prove to be more effective.

\section{Why a tourist village} % (fold)
\label{sec:wtv}

The `Tourist Village' conceived as a prototype will be a residential complex for the tourists and working space for craftsmen. Here, the emphasis will be on the environment. It will provide an opportunity for the man of city to get away from the pressures of urban life.

The village itself shall aim towards a sense of belonging, which can elevate cultural and traditional activities more than the modern structures, where people are aloof and isolated. The village carries the spirit of community and neighbourhood. The traditional architecture and the environment of this village i.e., learning together with local people can support and enhance in cultural growth, aside from the pleasant experience for the tourists.

The ethnic restaurant shall serve both the tourist as well as outsiders.

An open air theatre shall be so designed that the major landmark of Bhuj i.e., 'cenotaph complex' can be taken advantage of as a background. Here the light and sound show can be a memorable event for the tourist.

\begin{table}[H]
\caption{Spaces and areas of the design solution}
\label{tab:sads}
\centering
\small
  \begin{tabular}{+l^r^r}
  \rowstyle{\itshape}
    Spaces & Area (m$^2$) each & Total (m$^2$) \\
    \textsc{Residential} \\
    \hline
    Single occupancy rooms, 8~{\textnumero}s & 20 & 160 \\
    Double occupancy rooms, 8~{\textnumero}s & 25 & 200 \\
    Suites, 4~{\textnumero}s & 40 & 160 \\
    Dormitory, 2~{\textnumero}s & 120 & 240 \\
    Gymnasium & & 50 \\
    Games room & & 50 \\
    \textsc{Workshops} \\
    \hline
    Terracotta toys (excl outdoor space) & & 30 \\
    Murals (excl outdoor space) & & 30 \\
    Bandhani dying (excl outdoor space) & & 30 \\
    Embroidery & & 30 \\
    Block painting & & 30 \\
    Weaving & & 30 \\
    Leather embroidery & & 30 \\
    Knives carving & & 30 \\
    Mirror work & & 30 \\
    \textsc{Huts for artists} \\
    \hline
    Ladies, 2~{\textnumero}s & 40 & 80 \\
    Gents, 2~{\textnumero}s & 40 & 80 \\
    \textsc{art gallery} & & 100 \\
    \hline
    \textsc{Adminstration block} \\
    \hline
    Reception and information centre & & 15 \\
    Waiting lounge & & 30 \\
    Audio-visual room & & 30 \\
    Bathrooms & & 30 \\
    \textsc{Restaurants}, 2~{\textnumero}s & & \\
    \hline
    Ethnic style (excl outdoor) & & 100 \\
    Conventional style & & space \\
    \textsc{Open air theatre} \\
  \end{tabular}
\end{table}


% section wtv (end)

% chapter ds (end)