\chapter{Importance of this study} % (fold)
\label{cha:ios}

\section{The need for research} % (fold)
\label{sec:nfr}

``I do not want my house to be walled in on all sides and my windows to be stuffed. I want the cultures of all lands to be blown about my house as freely as possible. But I refuse to be blown off my feet by any.'' These words by Mahatma Gandhi (Young India, June 1, 1921) reflect the spirit which has moved us to meet in an atmosphere of cooperation and goodwill.

Tourism is not merely an exhibition, but an experience and hence the knowledge. The spice of tourism lies in understanding the people, by going deep into their ways of life. In developing an appreciation for other others' way of life and institution, tourism may create goodwill for a country. Tourism is the production of a nation's culture which has been defined as, ``The sum total of its achievements, expression of its own personality, its way of thinking, and acting.''

Tourist is a mirror to the country he belongs-to as much as they reflect social conditions prevailing in his motherland. He comes in contact with the people, which assists in developing an understanding, and brings closer to the community.

In the present world, however, commercialism is causing tourism to deviate from its goal of cultural exchange. Due to this, i.e., the spiritual and cultural growth of societies and the universal feeling of togetherness is drifting apart.

The problem of isolation between tourists and the local masses has reached an acute stage, which needs efficient remediation so that the basic theme behind cultural tourism is not adversely affected.

% section nfr (end)

\section{Present scenario} % (fold)
\label{sec:psc}

The growing commercialism is compartmentalising tourism, also known as `packaged tour', with an assortment of pre-determined destinations, standardised routines and activities, with increasingly less room for impromptu experiences.

Following conversation with a Spanish tourist during my study tour to Kutch is quite self-explanatory about the present scenario of cultural tourism.

Q: What has made you visit this interior part of India?

A (A Spanish teacher, 24): I had come across some literature and references regarding the colourful arts and crafts of Kutch. And those were strong enough to create a desire in my mind to visit this place personally. But upon my visit to Bhuj, and after checking with tourist agencies, I have found that there are no programmes through which I can intermingle with these artists, and for instance, get a feel for their way of working. What most offer are standardised visits to some monuments, sites, museums and handicraft centres. But the basic purpose for which I have come here, to connect, understand, learn, and take back a piece of this uniqueness, unfortunately remains unfulfilled.

Such drawbacks in the present system kill the natural instinct in the mind of a tourist. A city dweller today wishes to spend his vacation in an atmosphere different than his everyday life and to get involved in the true life style of that region which gives an everlasting impression and maximum satisfaction of his tour.

There are thousands of places of interest in India. Though developed in the interest of tourism, the local man is very rarely benefited. His skills hardly reach-up to the tourist, due to intermediaries, who end up taking a lion's share of the value generated by local craftsmen.

Under pressure to generate a reliable income stream, artists are forced to divert themselves into other works or activities, often away from generational artistry. Lack of patronage is killing the craft. A direct access to craft and their marketplace via tourism has the potential to reverse these dangerous trends.

% section psc (end)

\section{Measures to improve this scene} % (fold)
\label{sec:measures}

\subsection{The concept of cultural tourism} % (fold)
\label{sub:concept_ct}

Due to growing commercialism, tourism of present days are faced just to see the places rather than enjoying and feel by participating in the way of local area, its cultural heritage, history, and the people.

The primary thing that needs to done is to help tourist get a better perspective and knowledge all the while having a good chance to explore the local culture and heritage and also share their joy with others.

For a healthy relationship between tourists and the destinations they visit, there must be a balance of give and take. Pure pleasure can become demanding and even abusive if not managed responsibly. The tourism industry needs to reevaluate its role and approach, especially considering how the influx of tourists can bring both positive and negative cultural aspects. In the past, the attitude was simply to let tourists come because they bring money, but this perspective is changing. Tourism should not be a one-way street where visitors merely observe performances and demonstrations. Rather, it should be an interactive experience that evokes the art within the individual, and brings out a positive response from both visitors and locals.

The concept of holiday's are changing. A tourist has to raise his heart for the native. The enjoyment factor is greater if leisure becomes synonymous to being useful to people and regrouping our energies for a systematic scientific approach. Just being gregarious is not enough. Exhibitory tourism is unhealthy for culture, since there is nothing left to replenish it.

% subsection concept_ct (end)

\subsection{True interpretation of the community and its culture} % (fold)
\label{sub:tic}

The inflow of tourists to any region mainly depends on what a region can offer. In this particular field, India has all the aspects sufficiently strong that attracts tourists - foreign as well as domestic.

The basic need is:

\begin{enumerate}
  \item To interpret and exhibit unique offerings methodically, and
  \item To present attractive imagery via good publicity
\end{enumerate}

Some impressions, which may be different from previous experiences are often left-out after visiting new destination(s). The attraction of any destination arises to a large extent from the image it presents. It is in fact the expression of all objectives, thoughts with which a person or group judges a particular object or a place.

From the perspective of development, imagery is of great importance, because it influences selection. The main task of tourist resort and of national or heritage tourist organisation is to define, promote, and advertise the most attractive imagery possible. It has to be:

\begin{enumerate}
  \item As original as practicable: tourist facilities cannot add to the uniform of the regions or site concerned but if not added properly. They may spoil the character and look imposed.
  \item Truthful: If original character is retained, the image should be a perfect reflection of it or of the additional sources made available.
  \item Capable of implementing at an appropriate cost relative to their attractiveness with the financial means of the corresponding market to be able to compete with other similar destinations.
\end{enumerate} 

% subsection tic (end)

\subsection{Encouraging local artists and craftsmen} % (fold)
\label{sub:encor}

The growing tourism results in better economic gain. By encouraging the arts and crafts of that particular region, the artists get better returns, and their art is conveyed far and wide via publicity by visitors.

% subsection encor (end)

% section measures (end)

\section{Contributory value of this research} % (fold)
\label{sec:cvalue}

Being an ancient civilisation, India has vast heritage reserves, and as a result, there is no shortage of places of interest. Through rediscovery and grass-root connectivity to local community with access to learning, cultural artefacts, and participatory tourism, both visitors as well as regional art industry can stand to benefit and gain patronage, help upliftment, and improve welfare. In essence, this research is but an attempt to shed light in encouraging this --- through policies, development, re-focus on offerings, etc.

% section cvalue (end)

% chapter ios (end)
