\chapter{Importance of this study} % (fold)
\label{cha:ios}

\section{The need for research} % (fold)
\label{sec:nfr}

Mahatma Gandhi (\emph{Young India}, June 1921) on being culturally open-minded:

\begin{quote}
I do not want my house to be walled in on all sides and my windows to be stuffed. I want the cultures of all lands to be blown about my house as freely as possible. But I refuse to be blown off my feet by any.
\end{quote}

Tourism is an experience, not a mere exhibition. Its essence is in understanding the people by going deep into their ways of life. In developing an appreciation for others' way of life and institution, tourism may yet create goodwill for a country. Tourism is the production of a nation's culture which has been defined as, ``The sum total of its achievements, expression of its own personality, its way of thinking, and acting.''

Tourist is a mirror of the country he belongs-to, just as he is a reflection of the social conditions of his motherland. Coming in contact with people assists in developing an understanding, and brings him closer to the community.

There are problems of course. For instance, commercialisation is deviating tourism away from cultural enrichment via this exchange --- one of its goals. As a result, the spiritual and cultural growth of societies and the universal feeling of togetherness \emph{bajo el mismo sol} (under the same sun) is drifting apart. Lacking meaningful interactions, the feeling of isolation between tourists and the locals is rife, and requires urgent and efficient remedies to return back to the premise of cultural tourism.

% section nfr (end)

\section{Present scenario} % (fold)
\label{sec:psc}

With an assortment of pre-determined destinations, standardised routines and activities, \emph{aka} the `packaged tour' is causing commercial compartmentalisation, which is resulting into less and less room for impromptu experiences. The following conversation with a tourist from Spain during my study tour to Kutch is but an example.

Q (Author): What has made you visit this interior part of India?

A (Teacher from Spain, 24): I came across some literature and references regarding the colourful arts and crafts of \emph{Kutch}. And this created a strong desire to visit this place in person. But upon my visit to \emph{Bhuj}, and after enquiring with the agencies here, I have found that there are no programmes through which I can see and join these artists, and get a sense of their ways of working. What most offer are standardised visits to some monuments, sites, museums, and handicraft centres. But the main purpose of my visit is to connect, understand, learn, and take back a piece of this uniqueness, which unfortunately remains unfulfilled.

Such drawbacks kill natural instincts and the willingness to visit. A city dweller today wishes to spend the time in an atmosphere different from that of his everyday life and to get involved in the lifestyle of that region, which offers lasting memories and a sense of fulfillment from his tour.

There are thousands of places of interest in India. Though developed with tourism in mind, the local craftsman, limited in his exposure to the tourist, rarely reaps benefits from his artform, due to intermediaries, who end up taking a lion's share of the value generated.

Under pressure to generate a reliable income stream, artists are forced to divert themselves into other works or activities, often away from generational artistry. Lack of patronage is killing the craft. A direct access to craft and their marketplace via tourism has the potential to reverse these dangerous trends.

% section psc (end)

\section{Measures to improve this scene} % (fold)
\label{sec:measures}

\subsection{The concept of cultural tourism} % (fold)
\label{sub:concept_ct}

Heritage, history, and humanities is the essence of the concept of cultural tourism. It is but a feeling, a fragile one at that, and it can be easily destroyed when the experience of aiding intrigue, inquiry, and elation is turned into the rote and the routine. The vanity of a place is under threat when local craft shops are either overwhelmed or completely replaced by well-known brands as an example. When a tourist can purchase products from these anywhere, why would he bother to do it this particular place, and what connection would it offer to him when returns home?

Taking home a piece of something unique, be it a feeling, an experience, a performance, or a hand-made product, is what transcends its culture outside its geographical boundaries. When narrated, shared, or felt, they further spread the word communicating their intrigue and generate desire.

For a healthy relationship between tourists and the destinations they visit, there must be a balance of give and take. Pure pleasure can become demanding and even abusive if not managed responsibly. The tourism industry needs to reevaluate its role and approach, especially considering how the influx of tourists can bring both positive and negative cultural aspects. In the past, the attitude was simply to let tourists come because they bring money, but this perspective is changing. Tourism should not be a one-way street where visitors merely observe performances and demonstrations. Rather, it should be an interactive experience that evokes the art within the individual, and brings out a positive response from both visitors and locals.

The concept of a holiday is changing. A tourist has to raise his heart for the native. The enjoyment factor is greater if leisure becomes synonymous to being useful to people and regrouping our energies for a systematic scientific approach. Just being gregarious is not enough. Exhibitory tourism is unhealthy for culture, since there is nothing left to replenish it.

% subsection concept_ct (end)

\subsection{True interpretation of the community and its culture} % (fold)
\label{sub:tic}

The inflow of tourists to any region mainly depends on what a region can offer. In this particular field, India has all the aspects sufficiently strong that attracts tourists - foreign as well as domestic.

The basic need is:

\begin{enumerate}
  \item To interpret and exhibit unique offerings methodically, and
  \item To present attractive imagery via good publicity
\end{enumerate}

Some impressions, which may be different from previous experiences are often left-out after visiting new destination(s). The attraction of any destination arises to a large extent from the image it presents. It is in fact the expression of all objectives, thoughts with which a person or group judges a particular object or a place.

From the perspective of development, imagery is of great importance, because it influences selection. The main task of tourist resort and of national or heritage tourist organisation is to define, promote, and advertise the most attractive imagery possible. It has to be:

\begin{enumerate}
  \item As original as practicable: tourist facilities cannot add to the uniform of the regions or site concerned but if not added properly. They may spoil the character and look imposed.
  \item Truthful: If original character is retained, the image should be a perfect reflection of it or of the additional sources made available.
  \item Capable of implementing at an appropriate cost relative to their attractiveness with the financial means of the corresponding market to be able to compete with other similar destinations.
\end{enumerate} 

% subsection tic (end)

\subsection{Encouraging local artists and craftsmen} % (fold)
\label{sub:encor}

The growing tourism results in better economic gain. By encouraging the arts and crafts of that particular region, the artists get better returns, and their art is conveyed far and wide via publicity by visitors.

% subsection encor (end)

% section measures (end)

\section{Contributory value of this research} % (fold)
\label{sec:cvalue}

Being an ancient civilisation, India has vast heritage reserves, and as a result, there is no shortage of places of interest. Through rediscovery and grass-root connectivity to local community with access to learning, cultural artefacts, and participatory tourism, both visitors as well as regional art industry can stand to benefit and gain patronage, help upliftment, and improve welfare. In essence, this research is but an attempt to shed light in encouraging this --- through policies, development, re-focus on offerings, etc.

% section cvalue (end)

% chapter ios (end)
