\chapter{Preservation of culture} % (fold)
\label{cha:poc}

Culture is defined in many ways. It is (a) the art and other manifestation of human intellectual achievement regarded collectively; (b) a refined understanding of this intellectual development; (c) the customs, civilisation and achievement of a particular time or people; and (d) improvement by mental or physical training.

Thus culture is a precious possession of a civilisation. And various art forms constitute the soul of this culture. It is through various arts and traditions and crafts that any culture preserves its existence and flourishes.

\section{Present scenario} % (fold)
\label{sec:ps}

In the words of Pandit Jawaharlal Nehru, \emph{The discovery of India}:

\begin{quote}
  India is a geographical and economical entity, a cultural unity admits diversity, a bundle of contradictions held together by strong but invisible threads. Overwhelmed again and again, her spirit was never conquered.   

  About her there is the elusive quality of a legend long ago; some enchantment seems to have held in her mind, she is a myth and an idea, a dream and vision, and yet very real, present and persuasive.

  But however she changes that old witchery will continued and hold the hearts of her people. Though her attitudes may change, she would continue as of old and her store of wisdom will help her to hold on to what is true and beautiful and good in this harsh vindictive and graspic world.
\end{quote} 

Each and every corner of this large country has its distinctive culture, identity and so many things to offer right from various times of performing arts to various arts of painting, weaving, carving, pottery, and what not. Every artist in his field tries to work hard and using his ancestral gifted skills or may be those achieved through his own efforts to carry over that art form. Through this he also preserves his culture. But mainly he is doing that to earn his livelihood.

Great designer Ettore Sottsass says about Indian craftsmen:

\begin{quote}
  Indian craftsmen are open minded and totally concentrated on the tiny operations they do with their hands, on the silent, magic, surprising rituals performed by their finger, but on the contrary, they are also unaware and removed from the storms, the temptations of the world.
\end{quote}

For every art and artist it is the most important thing whether it gets some patronage or not. The art which was in its golden form when it was having royal patronage is dying today due to lack of proper appreciation and the returns which is it worth of. Due to this reason, which an artist finds that his father doesn't earn the required, in-spite of so much of hard-work, naturally he finds some other way to work. Because it is his livelihood that it is more important. In this process, the craftsmen are increasingly losing touch with their own traditions in terms of materials, techniques, designs and aesthetics of their arts and crafts. Sudden changes caused by modern industrialisation is also one of the reasons behind it.

But today with rapid rise in living standard of middle class and higher middle cast, they have raised their hearts for these art objects. A man wants his drawing room to be adorned with a beautiful mural, white a lady wants her wardrobe to be fulfilled with traditional \emph{patolas} and \emph{paithanis}, her bedspread to be nicely embroidered and her dining to display various antique pieces.

But these changing trends are still not sufficient to rejuvenate the crafts industry and most of the artists are still there where they were. The reason behind this being the presence of a middle man or a middle agency. Such middle person provides the urban consumers with their requirements from the artists. But keeps the major share of profit for himself. Thus the artist gets much less than what he should really be getting.

% section ps (end)

\section{Adapting to changing trends} % (fold)
\label{sec:act}

Adaptability stands out as a defining trait of a skilled artisan. While machines may replicate an artist's work, the craftsman must stay one step ahead of automation. A \emph{zardozi} and \emph{sequins} craftsman once remarked, ``How can I disregard alterations in materials, tools, and applications? Traditional adornments were intended for elephants, kings, and palaces, but where are they now?'' The craftsman hungers for instant feedback, akin to the fashion industry, which pushes them to stay vigilant and innovative.

This holds true even today. Dwelling excessively on archaic techniques can prove hazardous. The contemporary scenario demands that artists grasp the buyer's preferences to ensure market success. Consequently, artisans must discover what buyers would purchase and, correspondingly, undertake innovative efforts. Ornate traditional \emph{ghagras} and jackets may not yield significant profits in today's market. However, if an artisan employs those embroidery techniques in salwar suits or bedspreads, they may attain better returns.

To achieve this, craftsmen must remain aware of current trends and sensitively adapt their traditional skills to modern applications.

% section act (end)

\section{Revitalisation of traditional art forms} % (fold)
\label{sec:rev}

Tourism development stimulates the revitalisation of traditional art and hosts and impetus new creations. It provides market that has helped to preserve traditional art forms and keep culture alive.

Tourism increases the demand of arts and crafts with the result there is an abundance of jewellery, baskets, handicrafts, silver works, pottery, etc. Explorations by new ethnic groups has resulted in the borrowing, modification and refinement and development of particular art farm of art.

In brief tourism leads to:

\begin{enumerate}[noitemsep]
  \item Continued production of art forms by tribes
  \item Production of different art forms
  \item Maintenance of strong symbolic value of the product
  \item Availability of trained craftsmen
  \item Improvement in quality and artistic designs of arts and crafts
  \item Production of art form based on traditional ancestral lifestyle
  \item Expression of culture through art forms
  \item Creation of employment is fabrication of arts and crafts
  \item Renaissance in the production of art forms by avoiding the trends of specialisation standardisation and simplification
\end{enumerate}

% section rev (end)

% chapter poc (end)
