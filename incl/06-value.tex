\section{Efforts to increase tourism value} % (fold)
\label{sec:ettv}

The attractiveness of any destination is often based on the image it presents --- the one which is conjured-up, in part from direct or related experiences, and partly from information and publicity. 

Consider the \emph{Angkor Wat}. 

Discovered, forgotten, and rediscovered time and again, making this a fascinating story. Antonio da Madalena, a Portugese Capuchin friar, was one of the first Western visitors to visit. He told Diogo do Couto the historian in 1589, that ``it is not possible to describe it with a pen, particularly since it is like no other building in the world. It has towers and decorations and all the refinements which the human genius can conceive of.'' Angkor Wat was a 12th century phenomenon, and it is incredible to imagine that was forgotten in under three centuries as a stranded ruin before its discovery, only to be forgotten again, until, Henri Mouhout, a young French naturalist and explorer rediscovered it, wrote about and sketched it in detail, who eventually died there in 1861. Some of his works made their way back posthumously and rekindled great interest. He wrote:

\begin{quote}
  One of these temples -- a rival to that of Solomon and erected by some ancient Michelangelo might take an honourable place beside our most beautiful buildings. It is grander than anything left to us by Greece or Rome, and presents a sad contrast to the state of barbarism in which the nation is now plunged.
\end{quote}

\noindent Or the example of the \emph{Minaret of Jam}. A towering structure in Ghur Provice, Afghanistan on the banks of river Hari. Its feature and position belies the state it is found today in, and offers a poor reflection of the civilisation it once oversaw. Its design featuring a double-helix spiral staircase within it no less. Not only was its discovery odd, but on how its potential designer was found is legendary too.

\begin{quote}
  The minaret of Jam is covered with geometric and floral brickwork and turquoise-glazed epigraphic bands. One of the many peculiarities of the minaret is its almost exclusive dependence on varieties of angular script at a time when cursive had been in common use for hundreds of years for monumental inscriptions in that region. The sole use of cursive is for the architect's name, or signature, one Ali ibn Ibrahim al-Nisaburi. This suggests that Ali, or his family, was from the eastern Iranian city of Nishapur. Now this is where it gets very interesting. Apart from its mines being the world's source of turquoise (copper aluminium phosphate) for almost 2,000 years, and hence the turquoise ceranuis on the minaret, Nishapur was the home of Omar Khayyam. He lived between 1048--1131CE, about 100 years before the minaret was built.
\end{quote}

\noindent The minaret's design is most likely attributed to Omar Khayyam, whom the world knows as a poet, but in reality was a brilliant mathematician.

Rediscoveries of places like these lost to the sands of time, despite being the pinnacle of greatness in their time, help popularise them.  

Therefore, for development, the image of a place is important, because choice of destination is often not made objectively, but by the image it projects. Emphasis on originality, characteristics, historical context, circumstances under which it was built or developed are all contexts that weave narrative interests and entice one to pay a visit.

% section ettv (end)
