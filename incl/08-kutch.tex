\chapter{Kutch} % (fold)
\label{cha:kutch}

This chapter explores Kutch from the point of view of this research.

Among India's twenty-five states, Gujarat is the home of some of the most impressive traditions. And Kutch, ensconced in Gujarat's Northwest is a microcosm of the country's vibrant cultural diversity. It is an ancient land with a rich history due to its strategic location on the historic route to India whence many invaders arrived. It takes its name from the word `Kutchua' meaning tortoise, possibly due to its shape, geographical characteristics, and topographical features. It is bordered on the north and east by the flat desert, locally known as the `Rann'. Rann is a unique feature in the world. The entire area is covered with a thick layer of salt mixed with sand, and perhaps because of it, there are no traces of vegetation in the area. Kutch is bounded by the Arabian Sea to the west and the south-west and by the Gulf of Kutch to the south. In the north, it meets the border of Sindh, which is now in Pakistan since India's partition in 1947. To the north-east it runs along Rajasthan. One the east and south-east it links-up with other parts of Gujarat. The southern part, beyond the Gulf, is known as Saurashtra or Kathiawar.

Kutch stands out in isolation among Gujarat's other districts by its colourful assortment of flora and fauna, and it is a land of contrasts. There is the sea and the desert, the arid and the panoramic, coexisting side by side. Its rich traditions take provenance from eighteen different tribes, cultures, and language zones, to fuel curiosity and attractions to all kinds of tourists.

\section{Potential for tourism} % (fold)
\label{sec:pot}

For the traveller in search of beauty and spiritual stimulation, Kutch offers a delectable fare in complete collaboration with nature. Towns like Bhuj, Anjar, and Mandvi offer serenity, while Koteshwar, Hajipur, Narayan Sarovar, and Matano appeal to the devout. The gulf is a breeding ground for flamingos and pelicans during winter and in the little Rann of Kutch, the rare Indian wild asses roam free. Exotic handicrafts are a sought-after products here. Above all, the region is inhabited by people with sharp features, colourful dresses and smiling faces. They are renowned for their temperate nature and warm hospitality.

Though this arid region may evoke visions of vast desert tracks, it is full of surprises and has a unique culture of its own. Deft fingers weave rich tapestries of the Kutchi life, bursting forth into a spectrum of colours, harmonious design compositions, patterned permutations as if to offset its austere surroundings. Like a kaleidoscope, Kutch changes every person who views it, and where the delights are many.

The tourism corporation of Gujarat (\textsc{tcgl}) has worked out a master-plan, in which there are seven circuits identified. The positioning and products of each circuit are worked out. The measures to add value to tourism spots and infrastructure improvement actions are delineated. A development plan for each destination is ready. Kutch is identified as circuit-3 in this master-plan.

[Image: Seven circuits]

\subsection{Places of interest} % (fold)
\label{sub:poi}

\subsubsection{Bhuj} % (fold)
\label{ssub:bhuj}

[Image: The walled city of Bhuj]

The walled city of Bhuj is Kutch's district headquarters. It is famous for the wide variety of handicrafts. These include Kutchi embroidery with mirror work, bandhani, hand-printed textiles, and sarees.

% subsubsection bhuj (end)

\subsubsection{Aina Mahal Palace} % (fold)
\label{ssub:amp}

[Image: Aina Mahal Palace]

The palace is the eighteenth century creation of the extravagant Rao Lakhpathji (1741--60). Legend has it that he sent a local craftsman Ramsingh Malam to Europe to perfect his skills in glass-making and iron-founding. He commissioned Malam to construct Aina Mahal with its hall of mirrors of Venetian glass. The hall has white marble walls covered with mirrors, gilded ornaments, and the floor is a pleasure pool lined with tiles, with a platform above it surrounded by a series of fountains operated by an elaborate system of pumps below a Venetian chandelier.

% subsubsection amp (end)

\subsubsection{Cenotaphs Complex} % (fold)
\label{ssub:ctc}

[Image]

Cenotaphs or Chattardi are built in red stones. Ravaged by successive earthquakes since 1819, they are in ruins. Of these, the one built for Rao Lakha in 1770 is the largest and the finest. Polygonal in shape with balconies, it features an intricately carved roof. Other interesting Cenotaphs are the ones built for Rao Rayadhan, Rao Desai, and Rao Pragmal. Essentially, these cenotaphs serve as memorial grounds for the royal family.

% subsubsection ctc (end)

\subsubsection{Lakhpath} % (fold)
\label{ssub:lakhpath}

[Image]

Located north-west of Bhuj at a distance of 170km and accessible only by road, Lakhpath is an abandoned town. It was a prosperous port, once upon a time, yielding one lac koris, and hence the name.\footnote{Lac is equal to 100,000, and kori was once a Kutch monetary unit.} This was short-lived, as Indus river changed course, the port dried-up. It is now a barren plain of limestone rock. The sea is still closeby. Fifty years ago, the town was home to hundreds of families involved in fishing and manual labour. Over the years, however, it has emptied out with most people leaving in search of more lucrative employment opportunities.

Lakhpath is surrounded by a fort, an irregular polygon, and defended by round towers. The walls are considerable in height. Not just in architecture, but the stark lonesome and a vast view of the deserted, crumbling houses make it a rare spectacle. There is a mausoleum, dedicated to the local family, whose ancestors were known to have super-natural powers. The tomb is made of brick stone, with four arched doors, and its outer walls are decorated with floral motifs. The inside walls are engraved with passages from the Koran. In addition, there are Gurudwaras and and numerous temples.

% subsubsection lakhpath (end)

\subsubsection{Kera} % (fold)
\label{ssub:kera}

Kera finds a place on the tourist map mainly due to the tenth century Shiva temple. In its original form, it must have been extremely beautiful, based on what little remains at present. It was damaged in the earthquake of 1819.

% subsubsection kera (end)

\subsubsection{Koteshwar} % (fold)
\label{ssub:koteshwar}

Koteshwar is at a short distance from Narayan Sarovar and about 200km from Bhuj. It is an ancient place for pilgrimage. The existing temple is situated on a high plinth overlooking the sea. And it has a lovely sunset point.

% subsubsection koteshwar (end)

\subsubsection{Museums} % (fold)
\label{ssub:museums}

Bhuj has two: the Aina Mahal and the Kutch museum. Maharao Madansinh founded the former in 1977 with its collection of miniature paintings, wood carvings and glass works. The museum itself is a period room of 16--17th century. It also houses royal households, clothes, and jewellery. The hunting trophies are housed in Pragmahal, which is located on the same grounds.

Another hall of exhibits, though not officially a designated museum, is the Sharad Baug Palace, which showcases Maharao's personal belongings. It has a marvellous collection of trophies and presentation articles from the British Government. The architecture and artifacts of both palaces reflect European influence extensively.

Housed in a fine old building near Hamirsar lake, the Kutch museum initially formed a part of the school of art established by Maharrao Khengar III in 1877. Built by the state engineer Sir Fergusson, after whom it was named, the museum remained a private exhibit for the Maharraos until India's independence.

% subsubsection museums (end)

\subsubsection{Mandvi} % (fold)
\label{ssub:mandvi}

Founded in 1581 and located southwest of Bhuj, Mandvi has a fortification worth twenty-five bastions. The place is a well known centre for the production of the country's crafts. It once resounded to the bustle of shipyard where Rao Godji (1760--78) built and maintained a fleet of four hundred ocean going vessels, including one that sailed to England and back in 1760.

Mandvi is also known for its beautiful beach. The sea seems calm and clean, and there is a shallow run of sand, free from gravel, and it is safe to swim.

The Rukmavati bridge built in 1883 is the longest existing of its kind. The Vijay Vilas, a magnificent summer palace at 20km from Mandvi is a private property. The grounds house the main palace, servant's quarters, a `chattardi' and have a mahal that offers a beautiful view of the sunset. The palace exhibits various pieces of furniture, which have been maintained as they where while in use.

% subsubsection mandvi (end)

\subsubsection{Narayan Sarovar} % (fold)
\label{ssub:ns}

At 210km from Bhuj, Narayan Sarovar is one of the five holy lakes for the Hindus. It hosts a temple complex, and is an assimulation of many Hindu divinities.

% subsubsection ns (end)

\subsubsection{Dhola Veera} % (fold)
\label{ssub:dv}

The archaeological excavation here have revealed an Indus Valley civilisation site, It is one of the oldest and largest sites in India.

% subsubsection dv (end)

\subsubsection{Banni} % (fold)
\label{ssub:banni}

It is famous for its small villages, which are great storehouses of rich textiles and handicrafts, especially embroideries, block printed fabric, leather footwear, wood carvings, and pottery manufactured by the local people. Their ornaments, clothes, utensils, and everything they use will make one feel as it one has stepped into a lifestyle museum.

% subsubsection banni (end)

\subsection{How and where} % (fold)
\label{sub:haw}

Bhuj, the district headquarters of Kutch, would be a convenient base to explore the region. Most of Bhuj and its sights can be explored on foot and at leisure. Autorickshaws are also aplenty.

% subsection haw (end)

% subsection poi (end)

% section pot (end)

% chapter kutch (end)
