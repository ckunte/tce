\section{Crafts Museum} % (fold)
\label{sec:ce}

[Image here: Pragati Maidan, Architect: Charles Correa]

This case is undertaken to study the approach to museum of traditional arts. This is a small village designed that offers glimpses of various cultures of states.

\subsection{Location} % (fold)
\label{sub:ce_loc}

Crafts museum is located in Pragati Maidan, New Delhi. Pragati Maidan is India's progressive showcase and its identity in various sectors. Crafts emporium is located in this area and it represents the royal heritage of India's past. Crafts village is accessible from both gates of the Maidan.

[Image here: Crafts museum location]

% section ce_loc (end)

\subsection{Concept} % (fold)
\label{sub:ce_concept}

The basic function of this village is to offer glimpses of traditions from across India. As it is the museum of traditional arts, it must reflect traditional architecture of its region. Typical Indian huts, roads, havelis, temples, step wells, courts, etc., are the main features of this village.

The museum is arranged as a sequence of spaces and courtyards, which are interconnected by internal streets. A large permanent collection of folk and tribal arts, crafts, and textiles is housed in low concrete structures modelled on village scale units. The building resists the notion of a museum by presenting an almost everyday experience that is both socially as well as artistically gratifying. The complex also contains a handicrafts village, where works are produced at scale.

With exhibition as its primary function, keeping only crafts from various regions in an enclosed museum would be deficient. To overcome this, an attempt is made in creating an environment, in which artists from various states come from various regions of India to practice, produce and profit from their arts and crafts here.

[Image 1: Artists at work]

In this atmosphere, artists feel at home.

[Image 2: Artists at work]

% section ce_concept (end)

\subsection{Planning} % (fold)
\label{sub:ce_planning}

Many activities are considered here, according to which requirements are fixed, such as work space, exhibition spaces, audio-visual rooms, village crafts space, various crafts, temporary exhibitions, amphitheatre, administrative office, etc.

The great Hindu and Buddhist temples of the past, such as those in Bali, Java, and Southern India were structures around an open-to-sky courtyards with ceremonial paths.

[Image of a courtyard here]

In a similar way, Crafts museum is organised around a central pathway with a sequence of spaces along the pedestrian spine. As one passes on this, the spine is punctured and other spaces are seen. One can either visit any one particular exhibition or alternatively progress through all sections in a continuous sequence.

[Image here]

One such is the central Festive Court, a space that is used during festivals. Work spaces are not enclosed, so that people can easily see artists working. Performing arts are done in open spaces too.

The museum is concerned with artists, craftsmen, and their invaluable skills, and therefore conducts all its activities for their welfare. It strives to provide a free atmosphere to the visiting craftsmen.

% section ce_planning (end)

\subsection{Features} % (fold)
\label{sub:ce_feat}

% section ce_feat (end)

A permanent collection of 25,000 art work of folk and tribal crafts and textile is housed in a barely visible concrete building. The architect's challenge was to provide on the one side a building for safe preservation and to display rare art objects, and on the other, to not let the visage of the building to be so imposing that it would belittle the humbler objects collected from village homes.

It is a flexible building, reflecting the essence of an Indian village street, affable, accommodating, informal, and active. It is also full of unforeseen, undefined, and unexpected activities.

[Image here: Reflection of a typical village street]

Many spaces are created to depict the characteristics of a village and its atmosphere. These are areas used for gatherings or for socialising. An elevated platform acts as a stage for performing artists. These are very informal covered or uncovered spaces.

Workers from across the country come here to showcase their talents and to get better exposure than would be possible perhaps in their native places. But people coming to this place are limited and the rates are set by the Government. So profit margins differ and artists receive unevenly.

[Image here: Visitor can get the feel of an art in true sense]

[Image here: Informal character is very well maintained indoor]

[Image here: As well as in outdoor spaces]

\subsection{Courtyard} % (fold)
\label{sub:ce_courtyard}

The concept of a courtyard stems from India's climatic condition, which is generally hot and dry. So internal courtyards enable light and vegetation (for cooling) to the surrounding buildings.

Various construction techniques used are brought from their respective villages.

[Image here: Terracotta horses framing the courtyard]

% section ce_courtyard (end)

\subsection{Village Complex} % (fold)
\label{sub:ce_vc}

It is a remnant of a temporary exhibition on the theme of rural India setup in 1972 and spread about four acres. The village complex comprises of fifteen structures representing village dwellings, courtyards, and shrines from Arunachal, Himachal, and Madhya Pradesh, Gujarat, Rajasthan, West Bengal, Tamil Nadu, Andaman and Nicobar islands, etc.

% section ce_vc (end)

\subsection{Conclusion} % (fold)
\label{sub:ce_conc}

The Crafts museum is not merely a product of a clever transition of an architect's initial concepts into a concrete reality but has a timeless quality about it --- like that of India itself, where tradition and modernity coexist, sometimes as a collage, and other times as transitions from the former into the latter. The building itself defies the concept of a museum but rather a living and breathing eco-system that keeps alive the tradition of Indian art and crafts.

% section ce_conc (end)

% section ce (end)